\documentclass{article}

\usepackage[top=1in,bottom=1in,left=1in,right=1in]{geometry}

\title{\bfseries Constitution of Student-Run Computing and Technology at 
George Mason University}
\date{Last updated April 30, 2016}
\author{\texttt{srct@gmu.edu}}

\setcounter{secnumdepth}{0}

\usepackage{hyperref}

\usepackage{color}

\pagenumbering{gobble}

\begin{document}
  
  \maketitle
  
  %%%%%%%%%%%%%
  %%% SECTION 1: Name of Organization
  %%%%%%%%%%%%%
  \section{Article I --- Name of Organization}
  This student organization shall be named Student-Run Computing and 
  Technology (SRCT). The website for SRCT shall be located at 
  \url{srct.gmu.edu}.
  
  %%%%%%%%%%%%%
  %%% SECTION 2: Purpose of Organization
  %%%%%%%%%%%%%
  \section{Article II --- Purpose of Organization}
  Student-Run Computing and Technology (SRCT) will seek to enhance 
  student computing at Mason. SRCT will focus on establishing and 
  maintaining systems which would provide specific services to the 
  general Mason community.
  
  \subsection{Open Source}
  SRCT is strongly committed to the principles of Free and Open Source
  Software, and it is recommended that software used by or created by SRCT be
  Free and Open Source. However, we do acknowledge that there are times when
  proprietary software is more practical or more appropriate, and so our
  committment to open source shall not be considered an absolute requirement.
  
  %%%%%%%%%%%%%
  %%% SECTION 3: Membership
  %%%%%%%%%%%%%
  \section{Article III --- Membership}
  Membership in this organization will not be restricted on the basis of 
  race, color, religion, national origin, disability, sexual 
  orientation, veteran status, sex, or age. 
  \\ \\
  Membership is open to all currently enrolled GMU students in good 
  academic standing with the university with a minimum cumulative GPA of 
  at least 2.5 who support the advancement of the organization's 
  principles.
  \\ \\
  SRCT members will be classified as either \textbf{contributors} or 
  \textbf{developers}. In this document, the term ``general membership''
  shall refer to the combined body of contributors and developers.
  
  \subsection{Contributors}
  Contributors are individuals interested in joining the SRCT community. 
  All new members of SRCT are initially considered contributors. These 
  members may not participate in official votes, nor will they have 
  access to SRCT project management positions. They may contribute to 
  SRCT projects under the guidance and supervision of a project 
  manager.
  \subsection{Becoming a Developer}
  To attain developer status, a contributor must demonstrate leadership skills,
  commitment to SRCT principles, and technical ability. To become a developer,
  a contributor (hereafter referred to as ``prospective developer'') should
  notify the Executive Board of their desire to become a developer and present
  a justification for gaining developer status. This justification should
  include a brief history of the prospective developer's contributions to the
  organization. Alternatively, a developer may nominate a contributor to the
  Board and provide a similar rationale. The Board will review the rationale
  and vote on whether to admit the prospective developer. If the Board votes
  against admitting the developer, then the developer body (including the
  members of the Executive Board) shall be provided with the prospective
  developer's rationale and will have their own chance to vote, conducted
  according to the rules of an official vote. If the developers vote to
  admit the prospective developer, they shall be admitted without prejudice.
  If the developers vote to not admit the prospective developer, then the
  prospective developer shall not be given developer status, but may re-apply
  in the future. No prospective developer may be voted on more than twice in
  one semester.

  \paragraph{Developer Selection Criteria}
  To be accepted as a developer, an individual must meet two of the
  following three criteria, with emphasis on the first two.

  \begin{itemize}
    \item Active contributions to SRCT software projects.
    \item Active leadership participation in SRCT community events.
    \item Active social participation in the SRCT community, including
    meeting or event attendance.
  \end{itemize}
  
  \subsection{Developers}
  Developers are individuals invested in the SRCT community. These 
  members have full voting rights and may request project management 
  positions. A project management position entails access to SRCT 
  infrastructure as required by their project, and as overseen by the
  System Administrator. Additionally, project management entails the
  responsibility to maintain project documentation and oversee project
  development. Developers may also contribute to SRCT projects under 
  the guidance and supervision of the project manager. To attain 
  developer status, a contributor must demonstrate leadership skills, 
  commitment to SRCT principles, and technical ability. Prospective 
  developers will be selected based on these qualifications by official 
  vote. Any prospective developer will not be voted on more than twice 
  per semester.
  Any developer who leaves the university, whether by graduating or otherwise,
  loses their developer status. Developers who graduate automatically become
  alumni members. Developers who leave the university without graduating
  do not automatically gain alumni status, but may request alumni status
  from the Executive Board, who will decide the matter by a majority vote.
  Developers who are expelled lose their developer status and may not request
  alumni status.
  \\ \\
  The following categories, \textbf{alumni members} and
  \textbf{honorary members}, are special categories of membership. The
  purpose of these categories is to allow non-GMU people to contribute to
  SRCT. Members of these categories shall be listed as members, but will
  not be counted in membership totals, and will not count as ``Members'' when
  this document refers to such.

  \subsection{Alumni Members}
  Alumni members are former developers or contributors who graduated
  from George Mason in good standing with SRCT. They may contribute to
  projects and offer advice, but they may not vote, hold leadership
  positions, or control SRCT resources.

  \subsection{Honorary Members}
  Honorary members are people who have been recognized for providing
  significant service, expertise, or other contributions to SRCT. They
  may be nominated by any contributor or developer and must be approved
  by a majority vote of the Executive Board. They may contribute to projects
  and offer advice, but may not vote, hold leadership positions, or control
  SRCT resources.

  \subsection{Revocation of Membership}
  In extraordinary circumstances, if a motion is brought and seconded at 
  a meeting for revocation of a developer's status, the SRCT general membership
  will be notified by email of the motion not less than five days before the
  following meeting. If a quorum of developers is present, an official vote 
  will be held by secret ballot, requiring a $\frac{3}{4}$ supermajority of 
  present developers. Although an official vote, all present contributors will
  be allowed to present arguments for or against revocation of developer
  status prior to the vote. 
  \\ \\
  Developer status will be automatically removed if suspended or expelled by 
  the university.
  
  %%%%%%%%%%%%%
  %%% SECTION 4: Officers
  %%%%%%%%%%%%%
  \section{Article IV --- Officers}
  Officers' terms are for two semesters; there are no term limits. All 
  officers will attend any training required by OSI. The 
  \textbf{Executive Board} is defined as the collection of all five
  officer positions.
  
  \subsection{President}
  The President of SRCT presides over all meetings; serves as 
  spokesperson for SRCT; acts as its main liaison to the Advisor, ITU, 
  and OSI; oversees the transition to next semester's officers, and 
  ensures SRCT fulfills its constitutional obligations.
  
  \subsection{Vice President}
  The Vice President assists the President of SRCT to the extent the 
  President requests, and assumes the responsibilities of the President 
  in the President's absence, resignation, or removal. The Vice 
  President oversees all voting.
  
  \subsection{Treasurer}
  The Treasurer keeps accurate records of any expenditures and 
  accounting as outlined by the Office of Student Involvement's 
  ``Fiscal Management Policies and Forms.''
  
  \subsection{Secretary}
  The Secretary takes minutes at each meeting and publishes them to 
  SRCT's website, keeps record of the membership status of SRCT members, 
  and maintains all other necessary records and files.
  
  \subsection{System Administrator}
  The System Administrator is responsible for maintaining the hardware 
  and software systems at SRCT's disposal. This includes project 
  management software, SRCT servers, and SRCT website maintenance. The
  System Administrator also directly oversees all project managers.

  \subsection{Removal of Officers}
  Two SRCT developers may present a motion to remove an officer. The general
  membership will be notified by email, and developers will have no fewer
  than five days to review the motion. If a quorum is present at the next
  SRCT meeting, the motion may be approved by a $\frac{3}{4}$ vote by secret 
  ballot of all present developers. This is considered an official vote.
  
  \subsection{Advisor}
  The primary Advisor shall be a member of the faculty or staff at 
  George Mason University. The Advisor shall be selected by agreement of 
  the officers, and it is the responsibility of the officers to find a
  replacement should the Advisor no longer be suited for the position.
  The Advisor may offer guidance and support for SRCT, but may not 
  participate in any votes.
  
  %%%%%%%%%%%%%
  %%% SECTION 5: Elections
  %%%%%%%%%%%%%
  \section{Article V --- Elections}
  \subsection{Standard Elections}
  Elections will be held during the first meeting of March, with the
  results to come into effect at the start of the following semester.
  \\ \\
  Voting shall be handled in a secret ballot to be counted by the 
  highest ranking executive not in contest for any officer position, or if
  all positions are in contention, then by a quorum of all developers
  not directly involved in elections.
  \\ \\
  In the event of a tie, a second round of voting will be held between 
  the top two candidates following the same procedures of the first 
  round.
  \\ \\
  In the event of a further tie, the Advisor will determine the winner.
  
  \subsection{Special Elections}
  In event of the resignation of an officer, removal of an officer, or 
  any other situation in which there is a vacant executive position,
  an election will be held to fill that position for the remainder of 
  the term following the procedures above.

  %%%%%%%%%%%%%
  %%% SECTION 6: Impeachment or Resignation
  %%%%%%%%%%%%%

  \section{Article VI --- Impeachment or Resignation}
  While an officer is in an elected position, they must fulfill the 
  responsibilities of that position. If an elected officer fails to 
  perform their responsibilities or abuses the privileges of their 
  position, they will be subjected to impeachment and a subsequent 
  removal of office.
  
  \subsection{Removal of Officers}
  Two SRCT developers may present a motion to remove an officer. The general
  membership will be notified by email, and developers will have no fewer
  than five days to review the motion. If a quorum is present at the next
  SRCT meeting, the motion may be approved by a $\frac{3}{4}$ vote by secret 
  ballot of all present developers. This is considered an official vote.

  \subsection{Officer Resignation}
  In the event that an elected officer no longer wishes to hold their 
  position, they shall announce their decision at the next SRCT Organizational 
  meeting. If they are not able to attend the meeting, then they are 
  responsible for finding another method to communicate this decision to 
  the other officers.

  \subsection{Vacant Positions}
  If an elected position becomes vacant due to an impeachment or 
  resignation, a special election will be held to fill that vacant 
  position as described in Article V.II.

  
  %%%%%%%%%%%%%
  %%% SECTION 7: Meetings
  %%%%%%%%%%%%%
  \section{Article VII --- Meetings}
  SRCT shall meet on a weekly basis. Meetings should be scheduled to 
  accommodate the greatest number of members. All members of the Mason
  community are welcome to attend meetings. Attendance by SRCT 
  contributors and developers is not mandatory, but highly recommended. 
  Consistent absence may be viewed as grounds for revocation of 
  developer status.
  \\ \\
  The Executive Board has the right to call its own private meetings at
  its discretion.
  \\ \\
  A quorum shall be defined to include at least two officers, and either 
  a simple majority of developers or seven total developers including at
  least two officers, whichever is less. On official votes, in the event
  of a tie, the deciding vote shall be cast by the presiding officer.
  \\ \\
  The President shall preside over all meetings. In the event of the 
  President's absence, the meeting shall be presided over by the highest
  ranking available officer.
  \\ \\
  The most recent edition of Robert's Rules of Order will guide meeting 
  procedure.
  \\ \\
  Meetings will consist of at least one of two possible discrete 
  sections, an \textbf{organizational} section and a 
  \textbf{development} section. What type of meeting is scheduled for a
  particular week will be made clear at the time of the meeting's 
  announcement.
  
  \subsection{Organizational Meeting}
  Organizational meetings are to present motions, discuss status of SRCT 
  initiatives and other matters of interest or concern to SRCT, vote on 
  germane resolutions, and other matters permitted by SRCT's officers. 
  All members of the Mason community are welcome to present topics of 
  discussion at meetings. Developers and contributors may vote on 
  non-official resolutions.
  
  \subsection{Development Meeting}
  Development meetings are not subject to the rigorous standards 
  outlined above. They are entirely optional meetings. Development 
  meetings will be used to provide developers and contributors with an 
  opportunity to meet in person and work together on projects in a 
  relaxed atmosphere. There are no time or attendance limits, 
  requirements, or expectations placed on development meetings.
  
  %%%%%%%%%%%%%
  %%% SECTION 8.1: Finance
  %%%%%%%%%%%%%
  \section{Article VIII.I --- Finance}
  No dues shall ever be required as part of SRCT membership.
  \\ \\
  This clause shall not be construed to restrict other fundraising 
  mechanisms.
  
  %%%%%%%%%%%%%
  %%% SECTION 8.2: Amendments
  %%%%%%%%%%%%%
  \section{Article VIII.II --- Amendments}
  Two developers may jointly propose an explicitly defined 
  constitutional amendment at a meeting. The general membership will be 
  emailed a copy of the language of the amendment and have no fewer than
  five days to review the proposed amendment. At the next SRCT meeting, if a
  quorum is present, the amendment may be approved by a $\frac{3}{4}$ 
  vote by secret ballot of all present developers. This is considered an 
  official vote.
  \\ \\
  The Office of Student Involvement must review all amendments in the 
  same matter as a completely new constitution.
  \\ \\
  Changes which do not affect the meaning of the text, such as updated 
  formatting and links, may be made at the discretion of the board.
  \\ \\
  While contributors may not directly propose amendments, they may convince
  developers to sponsor an amendment on their behalf. Developers should only
  sponsor an amendment if they agree with the changes to be made.
  
  %%%%%%%%%%%%%
  %%% SECTION 9: Ratification
  %%%%%%%%%%%%%
  \section{Article IX --- Ratification}
  This constitution shall become effective upon approval by a 
  $\frac{3}{4}$ vote of the developers, and a Student Involvement staff 
  member. This is considered an official vote.
  
\end{document}
