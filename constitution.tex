\documentclass{article}

\usepackage[top=1in,bottom=1in,left=1in,right=1in]{geometry}

\title{\bfseries Constitution of Student-Run Computing and Technology at 
George Mason University}
\date{Last updated September 12, 2013}
\author{\texttt{srct@gmu.edu}}

\setcounter{secnumdepth}{0}

\usepackage{hyperref}

\usepackage{color}

\pagenumbering{gobble}

\begin{document}
  
  \maketitle
  
  %%%%%%%%%%%%%
  %%% SECTION 1: Name of Organization
  %%%%%%%%%%%%%
  \section{Article I --- Name of Organization}
  This student organization shall be named Student-Run Computing and 
  Technology (SRCT). The website for SRCT shall be located at 
  \url{srct.gmu.edu}.
  
  %%%%%%%%%%%%%
  %%% SECTION 2: Purpose of Organization
  %%%%%%%%%%%%%
  \section{Article II --- Purpose of Organization}
  Student-Run Computing and Technology (SRCT) will seek to enhance 
  student computing at Mason. SRCT will focus on establishing and 
  maintaining systems which would provide specific services to the 
  general Mason community.
  
  %%%%%%%%%%%%%
  %%% SECTION 3: Membership
  %%%%%%%%%%%%%
  \section{Article III --- Membership}
  Membership in this organization will not be restricted on the basis of 
  race, color, religion, national origin, disability, sexual 
  orientation, veteran status, sex, or age. 
  \\ \\
  Membership is open to all currently enrolled GMU students in good 
  academic standing with the university with a minimum cumulative GPA of 
  at least 2.5 who support the advancement of the organization's 
  principles.
  \\ \\
  SRCT members will be classified as either \textbf{contributors} or 
  \textbf{developers}.
  
  \subsection{Contributors}
  Contributors are individuals interested in joining the SRCT community. 
  All new members of SRCT are initially considered contributors. These 
  members may not participate in official votes, nor will they have 
  access to SRCT project management positions. They may contribute to 
  SRCT projects under the guidance and supervision of a project 
  manager.
  
  \subsection{Developers}
  Developers are individuals invested in the SRCT community. These 
  members have full voting rights and may request project management 
  positions. A project management position entails access to SRCT 
  infrastructure as required by their project, and as overseen by the
  System Administrator. Additionally, project management entails the
  responsibility to maintain project documentation and oversee project
  development. Developers may also contribute to SRCT projects under 
  the guidance and supervision of the project manager. To attain 
  developer status, a contributor must demonstrate leadership skills, 
  commitment to SRCT principles, and technical ability. Prospective 
  developers will be selected based on these qualifications by official 
  vote. Any prospective developer will not be voted on more than twice 
  per semester.
  \\ \\
  The following categories, \textbf{alumni members} and
  \textbf{honorary members}, are special categories of membership. The
  purpose of these categories is to allow non-GMU people to contribute to
  SRCT. Members of these categories shall be listed as members, but will
  not be counted in membership totals, and will not count as "Members" when
  this document refers to such.

  \subsection{Alumni Members}
  Alumni members are former developers or contributors who graduated
  from George Mason in good standing with SRCT. They may contribute to
  projects and offer advice, but they may not vote, hold leadership
  positions, or cannot control SRCT resources.

  \subsection{Honorary Members}
  Honorary members are people who have been recognized for providing
  significant service, expertise, or other contributions to SRCT. They
  may be nominated by any contributor or developer and must be approved
  by a majority vote of the Executive Board. They may contribute to projects
  and offer advice, but may not vote, hold leadership positions, or control
  SRCT resources.
  
  \subsection{Revocation of Membership}
  In extraordinary circumstances, if a motion is brought and seconded at 
  a meeting for membership revocation, all SRCT members will be notified 
  by email of the motion not less than five days before the following 
  meeting. If a quorum is present, an official vote will be held by 
  secret ballot, requiring a $\frac{3}{4}$ supermajority of present 
  members. Although an official vote, all present contributors will 
  be allowed to present arguments for or against membership revocation 
  prior to the vote. 
  \\ \\
  Members will be automatically removed if suspended or expelled by the 
  university.
  
  %%%%%%%%%%%%%
  %%% SECTION 4: Officers
  %%%%%%%%%%%%%
  \section{Article IV --- Officers}
  Officers' terms are for one semester; there are no term limits. All 
  officers will attend any training required by OSI. The 
  \textbf{Executive Board} is defined as the collection of all five
  officer positions.
  
  \subsection{President}
  The President of SRCT presides over all meetings; serves as 
  spokesperson for SRCT; acts as its main liaison to the Advisor, ITU, 
  and OSI; oversees the transition to next semester's officers, and 
  ensures SRCT fulfills its constitutional obligations.
  
  \subsection{Vice President}
  The Vice President assists the President of SRCT to the extent the 
  President requests, and assumes the responsibilities of the President 
  in the President's absence, resignation, or removal. The Vice 
  President oversees all voting.
  
  \subsection{Treasurer}
  The Treasurer keeps accurate records of any expenditures and 
  accounting as outlined by the Office of Student Involvement's 
  ``Fiscal Management Policies and Forms.''
  
  \subsection{Secretary}
  The Secretary takes minutes at each meeting and publishes them to 
  SRCT's website, keeps record of the membership status of SRCT members, 
  and maintains all other necessary records and files.
  
  \subsection{System Administrator}
  The System Administrator is responsible for maintaining the hardware 
  and software systems at SRCT's disposal. This includes project 
  management software, SRCT servers, and SRCT website maintenance. The
  System Administrator also directly oversees all project managers.

  \subsection{Removal of Officers}
  Two SRCT members may present a motion to remove an officer. All 
  members will be notified by email, and have no fewer than five days to 
  review the motion. If a quorum is present at the next SRCT meeting,
  the motion may be approved by a $\frac{3}{4}$ vote by secret ballot 
  of all present members. This is considered an official vote.
  
  \subsection{Advisor}
  The primary Advisor shall be a member of the faculty or staff at 
  George Mason University. The Advisor shall be selected by agreement of 
  the officers, and it is the responsibility of the officers to find a
  replacement should the Advisor no longer be suited for the position.
  The Advisor may offer guidance and support for SRCT, but may not 
  participate in any votes.
  
  %%%%%%%%%%%%%
  %%% SECTION 5: Elections
  %%%%%%%%%%%%%
  \section{Article V --- Elections}
  \subsection{Standard Elections}
  Elections will be held during the first meetings of March and 
  November, with the results to come into effect at the start 
  of the following semester.
  \\ \\
  Voting shall be handled in a secret ballot to be counted by the 
  highest ranking member not in contest for any officer position, or if
  all positions are in contention, then by a quorum of all project 
  managers not directly involved in elections.
  \\ \\
  In the event of a tie, a second round of voting will be held between 
  the top two candidates following the same procedures of the first 
  round.
  \\ \\
  In the event of a further tie, the Advisor will determine the winner.
  
  \subsection{Special Elections}
  In event of the resignation of an officer, then an election will be 
  held to fill their position for the remainder of the term following 
  the procedures above.
  
  %%%%%%%%%%%%%
  %%% SECTION 6: Meetings
  %%%%%%%%%%%%%
  \section{Article VI --- Meetings}
  SRCT shall meet on a weekly basis. Meetings should be scheduled to 
  accommodate the greatest number of members. All members of the Mason
  community are welcome to attend meetings. Attendance by SRCT 
  contributors and developers is not mandatory, but highly recommended. 
  Consistent absence may be viewed as grounds for revocation of 
  membership.
  \\ \\
  The Executive Board has the right to call its own private meetings at
  its discretion.
  \\ \\
  A quorum shall be defined to include at least two officers, and either 
  a simple majority of members or seven total members including at least
  two officers, whichever is less. On official votes, in the event of a 
  tie, the deciding vote shall be cast by the presiding officer.
  \\ \\
  The President shall preside over all meetings. In the event of the 
  President's absence, the meeting shall be presided over by the highest
  ranking available officer.
  \\ \\
  The most recent edition of Robert's Rules of Order will guide meeting 
  procedure.
  \\ \\
  Meetings will consist of at least one of two possible discrete 
  sections, an \textbf{organizational} section and a 
  \textbf{development} section. What type of meeting is scheduled for a
  particular week will be made clear at the time of the meeting's 
  announcement.
  
  \subsection{Organizational Meeting}
  Organizational meetings are to present motions, discuss status of SRCT 
  initiatives and other matters of interest or concern to SRCT, vote on 
  germane resolutions, and other matters permitted by SRCT's officers. 
  All members of the Mason community are welcome to present topics of 
  discussion at meetings. Developers and contributors may vote on 
  non-official resolutions.
  
  \subsection{Development Meeting}
  Development meetings are not subject to the rigorous standards 
  outlined above. They are entirely optional meetings. Development 
  meetings will be used to provide developers and contributors with an 
  opportunity to meet in person and work together on projects in a 
  relaxed atmosphere. There are no time or attendance limits, 
  requirements, or expectations placed on development meetings.
  
  %%%%%%%%%%%%%
  %%% SECTION 7: Finance
  %%%%%%%%%%%%%
  \section{Article VII --- Finance}
  No dues shall ever be required as part of SRCT membership.
  \\ \\
  This clause shall not be construed to restrict other fundraising 
  mechanisms.
  
  %%%%%%%%%%%%%
  %%% SECTION 8: Amendments
  %%%%%%%%%%%%%
  \section{Article VIII --- Amendments}
  Two members may jointly propose an explicitly defined 
  constitutional amendment at a meeting. All members will be emailed a 
  copy of the language of the amendment and have no fewer than five days 
  to review the proposed amendment. At the next SRCT meeting, if a 
  quorum is present, the amendment may be approved by a $\frac{3}{4}$ 
  vote by secret ballot of all present members. This is considered an 
  official vote.
  \\ \\
  The Office of Student Involvement must review all amendments in the 
  same matter as a completely new constitution.
  \\ \\
  Changes which do not affect the meaning of the text, such as updated 
  formatting and links, may be made at the discretion of the board.
  
  %%%%%%%%%%%%%
  %%% SECTION 9: Ratification
  %%%%%%%%%%%%%
  \section{Article IX --- Ratification}
  This constitution shall become effective upon approval by a 
  $\frac{3}{4}$ vote of the membership, and the Assistant Director for 
  Recognized Student Organizations. This is considered an official vote.
  
\end{document}
